\documentclass{sp2015course}
\usepackage{datetime}
\usepackage{listings,lstautogobble}
\lstset{
  basicstyle=\ttfamily,
  columns=flexible,
}
\usepackage{hyperref}

\begin{document}
	\begin{center}\Huge{Read this File First.}\end{center}
	\begin{small}
		Last updated: \today~at \currenttime
	\end{small}

	\textbf{Please read all parts of this file. It outlines the best practice for keeping your self in sync and not messing things up.}
	\begin{enumerate}
		\item Put java script in /js, css in /css, resources like images and sounds in /resources. When putting something new in, err on the side of organization. For example, when I wanted to add an example image, I created the directory /resources/images/placeholders.
		\item Everything that is not required to run the website should go in the /MISC directory (except this README file).
		\item Keep the website modular. Put each section in a separate html file.
		\item Best practices involving git: Don't ever do a git command with -f. If you don't do that nothing will be ruined.

		Here is what you need to know regarding git: how to add (or track) files. How to commit changes. How to push and pull changes.

		Additional things you might want to know is how to create new branches, switch branches, checkout projects, and revert changes.

		But here's the basics: to ``save" your changes locally (save a version that will always be recoverable) first navigate to the iGEMCornellWebsite2016 directory, then type
		\begin{lstlisting}[autogobble=true]
			git add -A
		\end{lstlisting}
		and then
		\begin{lstlisting}[autogobble=true]
			git commit -a -m "A description of what you did goes here"
		\end{lstlisting}
		To send your changes to the server (after you set up git to link with github):
		\begin{lstlisting}[autogobble=true]
			git push
		\end{lstlisting}
		And to get new changes
		\begin{lstlisting}[autogobble=true]
			git pull
		\end{lstlisting}
		Check out \href{https://training.github.com/kit/downloads/github-git-cheat-sheet.pdf}{\underline{this cheatsheet}} for more.
	\end{enumerate}
\end{document}











